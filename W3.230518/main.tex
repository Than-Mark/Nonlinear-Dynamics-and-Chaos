\documentclass[10pt,aspectratio=43,mathserif,table]{beamer} 
%  设置为 Beamer 文档类型,设置字体为 10pt,长宽比为16:9,数学字体为 serif 风格
\batchmode

\usepackage{graphicx}
\usepackage{animate}
\usepackage{hyperref}
\usepackage{diagbox} % 表头斜线分区
% 导入一些用到的宏包
\usepackage{amsmath,bm,amsfonts,amssymb,enumerate,epsfig,bbm,calc,color,ifthen,capt-of,multimedia,hyperref}
\usefonttheme{serif}
\usepackage{mathptmx}
\usepackage{xeCJK} %导入中文包
%\setCJKmainfont{SimHei} %字体采用黑体  Microsoft YaHei
%\setmonofont{Courier New}
%\setCJKmainfont[AutoFakeBold = {2.15},ItalicFont={KaiTi}]{SimSun}
%\setCJKfamilyfont{xw}{STXinwei}

%\setsansfont{Microsoft YaHei}

%\setsansfont{Arial}


\usetheme{Berlin} %主题
%\usecolortheme{sustech} %主题颜色

\usepackage[ruled,linesnumbered]{algorithm2e}

\usepackage{fancybox}
\usepackage{xcolor}
% \usepackage{times}
\usepackage{listings}

\usepackage{booktabs}
\usepackage{colortbl}

\newcommand{\Console}{Console}
\lstset{ %
	backgroundcolor=\color{white},   % choose the background color
	basicstyle=\footnotesize\rmfamily,     % size of fonts used for the code
	columns=fullflexible,
	breaklines=true,                 % automatic line breaking only at whitespace
	captionpos=b,                    % sets the caption-position to bottom
	tabsize=4,
	commentstyle=\color{mygreen},    % comment style
	escapeinside={\%*}{*)},          % if you want to add LaTeX within your code
	keywordstyle=\color{blue},       % keyword style
	stringstyle=\color{mymauve}\ttfamily,     % string literal style
	numbers=left, 
	%	frame=single,
	rulesepcolor=\color{red!20!green!20!blue!20},
	% identifierstyle=\color{red},
	language=c
}


\definecolor{mygreen}{rgb}{0,0.6,0}
\definecolor{mymauve}{rgb}{0.58,0,0.82}
\definecolor{mygray}{gray}{.9}
\definecolor{mypink}{rgb}{.99,.91,.95}
\definecolor{mycyan}{cmyk}{.3,0,0,0}

%题目,作者,学校,日期
\title{Uniform Oscillator}
%\subtitle{\fontsize{9pt}{14pt}\textbf{跨临界分岔}}
\author{Introducer: Yichen Lu\quad \newline  \newline \quad }
\institute{\fontsize{8pt}{14pt}}
\date{\today}
\newcommand{\concept}{Uniform Oscillator}

%学校Logo
%\pgfdeclareimage[height=0.5cm]{sustech-logo}{sustech-logo.pdf}
%\logo{\pgfuseimage{sustech-logo}\hspace*{0.3cm}}

\AtBeginSection[]
{
	\begin{frame}<beamer>
	\frametitle{\textbf{Contents}}
	\tableofcontents[currentsection]
\end{frame}
}
\beamerdefaultoverlayspecification{<+->}
% -----------------------------------------------------------------------------
\begin{document}
% -----------------------------------------------------------------------------

\frame{\titlepage}

% \section[Contents]{}   %目录
% \begin{frame}{Contents}
% \tableofcontents
% \end{frame}

% -----------------------------------------------------------------------------
\section{Concept}

\begin{frame}{\concept}
A point on a circle is often called an angle or a \textbf{phase}. Then the simplest oscillator of all is one in which the phase $\theta$ changes uniformly:
	\begin{itemize}
		\item $\dot{\theta}=\omega$, where $\omega$ is a constant. 
		\item The solution is $\theta(t)=\omega t+\theta_0$. 
		\item which corresponds to uniform motion around the circle at an angular frequency $\omega$. This solution is \textbf{periodic}, in the sense that $\theta(t)$ changes by $2\pi$ , and therefore returns to the same point on the circle, after a time $T = 2\pi / \omega$. We call T the period of the oscillation.
	\end{itemize}
\end{frame}

\section{Exercises}

\begin{frame}{Exercises 4.2.1}

\begin{itemize}
	\item (Church bells) The bells of two different churches are ringing. One bell rings every 3 seconds, and the other rings every 4 seconds. Assume that the bells have just rung at the same time. How long will it be until the next time they ring together? Answer the question in two ways: using common sense, and using the method of Example 4.2.1.
	\item Common sense: the first time they ring together will be after 12 (the least common multiple of 3 and 4) seconds.
\end{itemize}

\end{frame}

\begin{frame}
	
	\begin{itemize}
		\item (Church bells) The bells of two different churches are ringing. One bell rings every 3 seconds, and the other rings every 4 seconds. Assume that the bells have just rung at the same time. How long will it be until the next time they ring together? Answer the question in two ways: using common sense, and using the method of Example 4.2.1.
		\item Use the method of Example 4.2.1:
		$$
		T_{\mathrm{lap}}=\cfrac{2\pi}{\omega_1 - \omega_2} 
		= \left(\cfrac{1}{T_1} - \cfrac{1}{T_2}\right)^{-1} 
		= \left(\cfrac{1}{3} - \cfrac{1}{4}\right)^{-1}
		= \left(\cfrac{1}{12}\right)^{-1} = 12
		$$
	\end{itemize}

\end{frame}


\begin{frame}
	\LARGE \centering Thank you for listening!
\end{frame}

\end{document}

